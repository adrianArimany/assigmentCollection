\begin{assignment}{DesignA}
\addcontentsline{toc}{chapter}{Formativa 1: conjuntos y logica}

\duedate{Julio 15, 2025}
\useDesignAactivate

\subtitle{Relaciones y Órdenes}

\begin{problema}
$\forall A, \emptyset \subseteq A$
\end{problema}
\begin{demostracion}
Considere que $A$ es un conjunto, por el Axioma 0, existe el $\emptyset$.

Entonces por teorema del vació, tenemos que 

$$
\forall x , (x \notin \emptyset)
$$
es un enunciado es falso, lo que la implicación es verdadera. Entonces,
$$
\forall A (\emptyset \subseteq A)
$$
\end{demostracion}



\begin{demostracion}[Intento de demostración (incorrecta)]
Queremos demostrar que \( \forall A\,(\emptyset \subseteq A) \). \\

Se afirma que por el teorema anterior, \( \forall x\, (x \notin \emptyset) \) es \textbf{falsa}.\\

\textbf{(Error 1)}: En realidad, \( \forall x\, (x \notin \emptyset) \) es \emph{verdadera} por definición del conjunto vacío. \\

Luego se dice que como la hipótesis es falsa, entonces la proposición 
\[
x \in \emptyset \Rightarrow x \in A
\]
es verdadera. \\

\textbf{(Error 2)}: Aunque esta implicación es vacuamente verdadera, no se ha aplicado la definición de subconjunto, que es:
\[
\emptyset \subseteq A \iff \forall x\,(x \in \emptyset \Rightarrow x \in A)
\]

Además, no se justifica por qué esta proposición sería verdadera para \textbf{todo} \( A \).\\

\textbf{(Error 3)}: No se construye una demostración lógica general que funcione para cualquier conjunto \( A \), y no se introduce explícitamente la cuantificación universal sobre \( A \).\\

Por lo tanto, esta ``demostración'' no es válida. 
\end{demostracion}


\end{assignment}
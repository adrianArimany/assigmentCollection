\begin{assignment}{DesignA}
    

\addcontentsline{toc}{chapter}{Extra 1: Fallacy with the Separation Axiom}
\duedate{today}
\useDesignAactivate
\subtitle{Extra: Fallacy with the Separation Axiom}

\begin{problema}
Define the extension Axiom as:

$$
\exists B \forall x ( x\in B \leftrightarrow [x \in A \land \phi(x)]) 
$$
We will show that this version of the axiom is vulnerable to Russel Paradox.
\begin{demostracion}
In the defined extension axiom, the set $A$, is not universally define, this has the consequence that this $A$ must exist to "well-define" $\phi(x)$. Additionally, it requires that to construct set B, one must determine a good A first. 

Consider the following case, where A is a no bounding set. Basically how naive set theory defines sets:
$$
A = \{x \mid \phi(x)\}
$$

Now let $\phi(x) := x \notin x$, so from the extension axiom:
$$
\forall R \forall x (x \in R \leftrightarrow x \notin x)
$$

Here R would be $R = \{ x \mid x \notin x \}$.\\ 

So from Russel paradox, is $R \in R$? Which as we know leads to a double contradiction.\\

How to fix this, use the following version for the separation axiom or subset axiom:

$$
\forall A \exists S \forall x (x \in S \leftrightarrow (x \in A \land \phi(x))
$$

Notice that here the set A is universally defined, so one can't simply take any set A. So suppose that

$$
\phi(x) := x \notin x
$$

Now let $A = {\emptyset}$, then 

$$
S = \{ x \in \{\emptyset\} \mid x \in x\}
$$

Notice that if $x = \emptyset$, then the statement $\emptyset \notin \emptyset$ is true, so $S = \{\emptyset\}$ 

\end{demostracion}
\end{problema}

\end{assignment}
